\begin{enumerate}
	\item Describe how common and vastly used Scrum is.
	\item Describe what Scrum does, that works well in the industry.
	\item Describe what Scrum does, that DOESN'T works well in the industry. - Self-organised chaos. Process improvement on team level (bad retrospectives).
	\item Introduce Lean as solution. (Later: How will Lean deal with the well-functioning timeboxing vs. flow?)
\end{enumerate}

Scrum is currently the most used agile framework. According to the \textit{9th Annual State of Agile™ Survey} \citep{VersionOne}, 56\% of the companies that do agile development use Scrum and at least 16\% more use Scrum combined with something else.

The overall principles of Scrum were first introduced by \citet{Takeuchi1986}, formalised a few years later by \citet{Schwaber} and it has since been refined into what it is today.
They describe Scrum as a lightweight framework for managing complex product development. 

Scrum has become a success because it allows for flexibility as well as predictability.
Especially well-functioning aspects are, but not limited to:
\begin{itemize}
	\item Timeboxing - Using a timebox called “sprint”, Scrum focuses the attention to the next timebox without getting distracted by future tasks. This practice is widely used and liked by the industry, according to XXX.
	\item Planning - Scrum provides multiple planning activities and artefacts for the sprint planning as well as daily task planning. According to XXX Scrum planning activities are widely used with good results.
	\item ...
\end{itemize}

Although Scrum has several good aspects, some can prove to be counter productive in certain contexts.
 
The idea of self-organising teams was introduced in XXX due to the problems with the previous approach of XXX.
... many occurrences of self-organising teams turning to self-organising chaos are reported ...
... this problem is, according to XXX, often due to either failed implementation, misunderstood values, or poor/lack of leadership. ...

The aspect of process improvement on the team level are in Scrum managed by the retrospective meetings. These meeting, however, are reported by XXX, XXX, and XXX to often be poorly conducted, unfruitful, and a waste of time. 

... looked at frameworks which values process improvement on the team level ...
... Lean ... Lean may be an option for solving some of the shortcomings of Scrum. ...



Plan 06-10-2015:
\begin{enumerate}
	\item Complete writing introduction.
	\item Find sources for introduction.
	\item Read Grounded Theory and SLR research method.
	\item Create slides with comparison of GT vs. SLR.
	\item Write report section about research method.
	\item Create slides for introduction.
	\item Potentially rewrite/add to report section introduction.
\end{enumerate}
