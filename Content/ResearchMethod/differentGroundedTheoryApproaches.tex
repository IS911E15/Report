\section{Different approaches to/versions of Grounded Theory}
As with other research methods, there exists multiple approaches to Grounded Theory.
The most common approaches are written by:
\begin{enumerate}
	\item Glaserian Grounded Theory \footnote{Glaser BG, Strauss A (1967) The Discovery of Grounded Theory: Strategies for Qualitative Research. Aldine, Chicago Illinois}
	\item \footnote{Corbin J, Strauss A (1990) Grounded Theory Research: Procedures, Canons, and Evaluative Criteria. QualSociol 13(1):19} and \footnote{Strauss A, Corbin J (1998) Basics of Qualitative Research: Techniques and Procedures for Developing  Grounded Theory. Sage Publications, Second Edition}
	\item \footnote{Charmaz K (2006) Constructing Grounded Theory: A Practical Guide Through Qualitative Analysis. Thousand Oaks, Sage}
\end{enumerate}

...


...


Although these three approaches are different, we decided to follow the CORBIN STRAUSS approach, since it was the most common ADOLF 2011 and we had some materials available. Further discussion will not be pursuit, adhering to one of the guidelines recommended by ADOLF 2008, which states “Do not get caught up in the debate over which method is the “best” Grounded Theory method. The best method is the one that has the most resource and support availability. Nonetheless, be clear which method you are employing in your study”.
















Classical/Glaserian Grounded Theory (Glaser and Stauss 1967) (Adolf)

Formalised and procedure-lised Grounded Theory (Strauss and Corbin 1990) (Wolfwinkel)

Constructing Grounded Theory (Charmaz 2006)