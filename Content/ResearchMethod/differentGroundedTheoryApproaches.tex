\section{Different approaches to/versions of Grounded Theory}
\citet{Adolph-2011-GroundedTheory} recommends \textit{“Do not get caught up in the debate over which method is the “best” Grounded Theory method. The best method is the one that has the most resource and support availability. Nonetheless, be clear which method you are employing in your study”}.

Therefore we looked at the different approaches to Grounded Theory and identified the three most common approaches.
\begin{itemize}
	\item Glaserian/Classic Grounded Theory (CGT) based on \citet{Glaser-1967-GroundedTheory} and \citet{Glaser-1978-GroundedTheory}
	
	
	\item Straussian Grounded Theory / Qualitative Data Analysis (QDA)
	\footnote{Corbin J, Strauss A (1990) Grounded Theory Research: Procedures, Canons, and Evaluative Criteria. QualSociol 13(1):19}
	 and 
	 \footnote{Strauss A, Corbin J (1998) Basics of Qualitative Research: Techniques and Procedures for Developing  Grounded Theory. Sage Publications, Second Edition}
	
	\item Constructivist Grounded Theory
	\footnote{Charmaz K. (2000). Grounded theory: Objectivist and constructivist methods. In N. Denzin,  \& Y. Lincoln, (eds.), Handbook of Qualitative Research (pp. 509-535). Thousand Oaks, CA, Sage Publications, Inc.}
	 and 
	\footnote{Charmaz K (2006) Constructing Grounded Theory: A Practical Guide Through Qualitative Analysis. Thousand Oaks, Sage}
\end{itemize}

We decides to follow the Straussian Grounded Theory, due to it being the more ...

...






