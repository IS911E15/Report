\begin{enumerate}
	\item Explain how there is no standard for literature studies in Information Systems.
	\item Describe limitation of articles.
	\item Describe limited problem domain.
	\item Describe SLR approach, pros. and cons.
	\item Describe Grounded Theory approach, pros. and cons.
	\item Compare SLR vs. Grounded Theory.
	\item Explain why XXX is our choice.
	
	
	
	\item What tool(s) did we use?
	\item How did we read the articles?
	\item What was the initial question and how did it evolve over time?
	\item 
\end{enumerate}


\section{What is theory?}
The term theory is, in the context of Grounded Theory, quite different than we, as engineers, are used to. We “regard a theory as a universal truth like Einstein’s theory of relativity”(Adolf 2011), but in this context “Social theories do not claim to be universal truths and vary in their scope and generalizability.”


Gregory



\section{Different approaches to Grounded Theory}
Classical/Glaserian Grounded Theory (Glaser and Stauss 1967)

Formalised and procedure-lised Grounded Theory (Strauss and Corbin 1990)

Constructing Grounded Theory (Charmaz 2006)

 


\section{Classical/Glaserian Grounded Theory}
Glaser B, Holton J (2004) Remodeling Grounded Theory. Forum Qualitative Sozialforschung/Forum:
Qualitative Social Research in Nursing \& Health 5(2), Retrieved from http://www.qualitative-research.net/
fqstexte/2-04/2-04glaser-e.htm





“How does a ScrumMaster incorporate/integrate Lean thinking?” Possible initial question?

“For anyone considering using the Grounded Theory method for the first time, we strongly
recommend a pilot study to practice and develop skills.” - Yay